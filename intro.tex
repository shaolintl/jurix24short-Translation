Artificial Intelligence is significantly transforming the legal field, including areas like legal analytics \cite{alschner2021ai}, client onboarding, market research \cite{martin2019marketing}, and other machine learning applications \cite{surden2021machine}.

Nevertheless, the use of Large Language Models (LLMs) in various legal applications suffer from two main drawbacks.
Foremost, the statistical approach behind the use of LLMs prevents accurate results and also impairs demonstrability \cite{cooper2022accountability} - two properties of high importance in the legal domain.
Second, LLMs are based on a textual representation, and while encoding techniques improve the semantic analysis of these texts \cite{vaswani2017attention},
the result is a still far cry from an equivalent work done by humans \cite{janatian2023text}.

In \cite{janatian2023text}, the authors have attempted to improve the formalization process by using GPT-4.
Their task was to extract a logical structure from a specific kind of legal sentences.
The logical structure considered is a flow diagram where nodes represent yes/no questions and edges positive and negative answers.
For this task, the authors focused on either stand-alone articles or paragraphs containing criteria and conclusions.
The experiments in the paper show promising results - $67.5\%$ have correctly captured the logical meaning of the text, while $40\%$
could be directly used in a formalization without further changes from humans.

One formalization challenge which was not dealt with in the above work was the identification of cross references between articles.
Such cross-references are very common in legislation and represent exceptions, definitions, further conditions, etc.

The on-going work described in this paper tries to continue the above effort and focuses on the following question

\begin{itemize}
  \item {\bf RQ:} Can an isomorphism-preserving process affect the quality of formalization?
\end{itemize}

To this aim, we have defined a formalization language and a process, which we believe, facilitate the automated formalization done by LLMs.

The first part of the paper focuses on the presentation of this language properties and the methodology taken. The second part presents a first evaluation of the method. We conclude with future steps towards automatic formalization of legal documents.
